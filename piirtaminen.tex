% Created 2019-04-22 Mon 23:00
% Intended LaTeX compiler: pdflatex
\documentclass[11pt]{article}
\usepackage[utf8]{inputenc}
\usepackage[T1]{fontenc}
\usepackage{graphicx}
\usepackage{grffile}
\usepackage{longtable}
\usepackage{wrapfig}
\usepackage{rotating}
\usepackage[normalem]{ulem}
\usepackage{amsmath}
\usepackage{textcomp}
\usepackage{amssymb}
\usepackage{capt-of}
\usepackage{hyperref}
\author{Reima}
\date{\today}
\title{}
\hypersetup{
 pdfauthor={Reima},
 pdftitle={},
 pdfkeywords={},
 pdfsubject={},
 pdfcreator={Emacs 26.2 (Org mode 9.1.9)}, 
 pdflang={English}}
\begin{document}

\section*{Piirtäminen}
\label{sec:org4ed876c}
\subsection*{Piirtoikkunan pisteen koordinaatit}
\label{sec:orgf67c38d}

Processing-ohjelmointikielessä piirretään kuvioita ja animaatioita
piirtoikkunaan. Piirtoikkunan eri kohtiin eli \emph{pikseleihin}
viitataan \(x\)- ja \(y\)-koordinaateilla. Esimerkiksi pisteen
\((70, 40)\) \(x\)-koordinaatti on 70 ja
\(y\)-koordinaatti 40. Toisin kuin matematiikassa, piirtoikkunassa
piste \((0, 0)\) on \emph{vasemmassa yläkulmassa} ja \(y\)-koordinaatin
arvot kasvavat \emph{alaspäin}. Alla oleva kuva esittää piirtoikkunaa,
jonka leveys on 600 ja korkeus 400. Kuvaan on piirretty myös
kaikkien ikkunan kulmapisteiden koordinaatit sekä pisteiden \((70,
   40)\) ja \((220, 130)\) kohdat.


\begin{verbatim}
usepackage ("inputenc", "utf8");
defaultpen (fontsize (14));
int LEVEYS = 600;
int KORKEUS = 400;
size (LEVEYS, KORKEUS);
filldraw (xscale (LEVEYS) * yscale (KORKEUS) * unitsquare, lightgray);
// muunnos Processing-koordinaatistosta perinteiseen koordinaatistoon
transform T = shift ((0, 400)) * yscale (-1); 
real s = -.05 * LEVEYS; // akselien translaatio pois ikkunan reunalta
draw ("$x$", T * shift ((0, s)) * ((0, 0) -- (LEVEYS, 0)), LeftSide, EndArrow);
draw ("$y$", T * shift ((2*s, 0)) * ((0, 0) -- (0, KORKEUS)), EndArrow);

pair A = (70, 40);
pair B = (220, 130);

for (pair P : new pair[] {A, B})
  dot ((string) P, T * P, NE);
for (int x : new int[] {0, LEVEYS})
  for (int y : new int[] {0, KORKEUS})
    {
      pair P = (x, y);
      dot ((string) P, T * P, (x == 0 ? W : E));
    }
\end{verbatim}

Pistettä \((0, 0)\) kutsutaan piirtokoordinaatiston \emph{origoksi}. 

Eräs Processing-kielen peruskuvioista on suorakulmio (englanniksi
\emph{rectangle}). Seuraavassa esimerkissä piirretään suorakulmio
ikkunaan, jonka koko on \(600\times 400\) pikseliä. Piirtoikkuna on
väriltään harmaa, ja piirretyssä valkoisessa suorakulmiossa on
musta reunus.
\begin{verbatim}
size (600, 400);
rect (70, 40, 150, 90);
\end{verbatim}

Lauseke \texttt{size (600, 400);} avaa piirtoikkunan, jonka leveys on 600
pikseliä ja korkeus 400 pikseliä. Lauseke \texttt{rect (70, 40, 150, 90);}
piirtää suorakaiteen, jonka vasen yläkulma on kohdassa \((70, 40)\)
ja jonka leveys on 150 ja korkeus 90. Koska suorakaiteen yläkulman
\(x\)-koordinaatin ja leveyden summa on \[ 70 + 150 = 220, \] ja
yläkulman \(y\)-koordinaatin ja korkeuden summa on \[ 40 + 90 =
   130, \] suorakaiteen oikea alakulma on samassa kohdassa kuin yllä
olevaan kuvaan piirretty piste \((220, 130)\).

Tarkalleen ottaen pikselit ovat pieniä neliöitä, mikä vaikuttaa
hieman oikean alakulman tarkkoihin koordinaatteihin. Tähän palataan
kappaleessa \href{sisakkaiset-silmukat.org}{sisäkkäiset silmukat}.
\subsection*{Ohjelmakoodin muoto}
\label{sec:org8eda858}
Tietokoneohjelma koostuu tyypillisesti \emph{ohjelmariveistä}, joita
edellä olevasta ohjelmasta löytyy kaksi. Processing-ohjelma on
hyvin tarkka monista ohjelmarivin yksityiskohdista. Jotkut asiat
ohjelmoija voi itse päättää.

Edellä olevasta ohjelmasta voidaan jättää pois kaikki tyhjä tila
eli välilyönnit, ja ohjelma toimii silti samalla
tavalla. Välilyöntien käyttäminen on siis tyylikysymys. Myös tyhjiä
rivejä ja rivinvaihtoja voidaan lisätä ohjelmaan huoletta.

\begin{verbatim}
size(600,
     400);

rect(70,40,150,90);
\end{verbatim}

Monet muutokset johtavat siihen, että ohjelma ei joko toimi tai
lopputulos muuttuu. Ennen kaikkea
\begin{itemize}
\item suoritettavan komennon lopussa täytyy olla puolipiste eli \texttt{;}
\item sanat \texttt{size} ja \texttt{rect} täytyy kirjoittaa pienillä kirjaimilla
\item sulkujen \texttt{()} täytyy olla tavalliset sulut, ei esimerkiksi
aaltosulut \texttt{\{\}} tai hakasulut \texttt{[]}
\item sulkujen sisällä olevien lukujen täytyy olla eroteltuina
toisistaan pilkuilla
\item Processing-ohjelmassa komennon \texttt{size()} täytyy olla aina ohjelman
ensimmäinen rivi.
\end{itemize}

Kokeile tehdä muutoksia ohjelmaan ja katso, mitä tapahtuu kun yrität
suorittaa ohjelman.
\subsection*{Muita kuvioita}
\label{sec:org254e2e2}
Suorakaiteen piirtämiseen käytetään yllä \emph{funktiota}
\texttt{rect()}. Funktioiden nimiin viitataan usein laittamalla sulut
nimen perään, jotta tiedetään kyseeessä olevan nimenomaan
funktio. Tällöin sulkujen sisältö jätetään yksinkertaisuuden vuoksi
monesti tyhjäksi.

Funktio \texttt{rect()} piirtää suorakaiteen, jonka sivut ovat
piirtoikkunan sivujen suuntaiset. Muunlaisia nelikulmioita voidaan
piirtää funktiolla \texttt{quad()}, jonka nimi tulee englannin kielen
nelikulmiota tarkoittavasta sanasta \emph{quadrilateral}. Tällöin
sulkujen sisään laitetaan kaikkien neljän pisteen koordinaatit
järjestyksessä joko myötä- tai vastapäivään kuviota
kiertäen. Seuraavassa esimerkissä piirretään nelikulmio, jonka
kärkipisteet ovat myötäpäivään kiertäen \((70, 40)\), \((220, 40)\),
\((220, 130)\) sekä \((70, 130)\). Lopputuloksena saadaan
samankaltainen suorakulmio kuin edellisessä esimerkissä.

\begin{verbatim}
size (600, 400);
quad (70, 40, 220, 40, 220, 130, 70, 130);
\end{verbatim}

Kolmioita voidaan piirtää vastaavalla funktiolla \texttt{triangle()} jolle
annetaan kolmen kärkipisteen koordinaatit. 

Mikäli kuvioita piirretään useampia, seuraava piirretään edellisen
päälle. Seuraavissa esimerkeissä vain piirtokomentojen järjestys on
vaihdettu.

\begin{verbatim}
size (600, 400);
triangle (50, 50, 50, 350, 350, 350);
quad (50, 200, 200, 350, 350, 200, 200, 50);
\end{verbatim}

\begin{verbatim}
size (600, 400);
quad (50, 200, 200, 350, 350, 200, 200, 50);
triangle (50, 50, 50, 350, 350, 350);
\end{verbatim}
\subsection*{Kuvion suunnitteleminen}
\label{sec:orgbebab83}
Kuvioiden kärkipisteiden koordinaatit kannattaa usein suunnitella
kynällä ja paperilla ennen kuvan piirtämistä
tietokoneella. Esimerkiksi yllä oleva kahden kuvion kuva
suunniteltiin seuraavasti. Tavoitteena oli piirtää kuvan kaltainen
kuvio, jossa on 
\begin{itemize}
\item suorakulmainen tasakylkinen kolmio sekä
\item neliö, jonka kaksi kärkipistettä ovat kolmion kylkien
keskikohdissa.
\end{itemize}
Kolmion yhdeksi pisteeksi valittiin \((50, 50)\) ja kylkien
pituudeksi 300.

Suunnittelu voidaan yleensä tehdä useammalla eri tavalla. Tässä
tapauksessa kärkipisteet voidaan etsiä vaikkapa seuraavasti alla
olevan kuvan mukaisesti.
\begin{enumerate}
\item Piste \(A (50, 50)\) on annettu.
\item Pisteellä \(B\) on sama \(x\)-koordinaatti kuin pisteellä \(A.\)
Koska kolmion kyljen pituus on 300 ja \(y\)-akselin arvot
kasvavat alaspäin, pisteen \(B\) \(y\)-koordinaatti on 300
yksikköä suurempi kuin pisteen \(A\). Niinpä \(B(50, 350).\)
\item Pisteillä \(B\) ja \(C\) on sama \(y\)-koordinaatti, mutta pisteen
\(C\) \(x\)-koordinaatti on 300 suurempi. Täten \(C(350, 350).\)
\item Piste \(D\) on pisteiden \(A\) ja \(B\) puolivälissä, joten \(D(50,
      200).\) Vastaavasti \(E\) on pisteiden \(B\) ja \(C\) puolivälissä,
jolloin \(E(200, 350).\)
\item Koska \(DEFG\) on neliö, pisteellä \(F\) on sama \(x\)-koordinaatti
kuin pisteellä \(C\) ja sama \(y\)-koordinaatti kuin pisteellä
\(D,\) joten \(F(350, 200).\)
\item Vastaavasti pisteellä \(G\) on on sama \(x\)-koordinaatti kuin
pisteellä \(E\) ja sama \(y\)-koordinaatti kuin pisteellä \(A,\)
joten \(G(200, 50).\)
\end{enumerate}

\begin{verbatim}
usepackage ("inputenc", "utf8");
defaultpen (fontsize (14));
int LEVEYS = 600;
int KORKEUS = 400;
size (LEVEYS, KORKEUS);
filldraw (xscale (LEVEYS) * yscale (KORKEUS) * unitsquare, lightgray);
// muunnos Processing-koordinaatistosta perinteiseen koordinaatistoon
transform T = shift ((0, 400)) * yscale (-1); 
real s = -.05 * LEVEYS; // akselien translaatio pois ikkunan reunalta
draw ("$x$", T * shift ((0, s)) * ((0, 0) -- (LEVEYS, 0)), LeftSide, EndArrow);
draw ("$y$", T * shift ((2*s, 0)) * ((0, 0) -- (0, KORKEUS)), EndArrow);

pair A = (50, 50);
int s = 300; // sivun pituus
pair B = shift ((0, s)) * A;
pair C = shift ((s, 0)) * B;

draw (T * (A -- B -- C -- cycle), dashed);
pair D = midpoint (A -- B);
pair Ep = midpoint (B -- C);
pair F = (C.x, D.y);
pair G = (Ep.x, A.y);
draw (T * (D -- Ep -- F -- G -- cycle), dashed);
pair[] points = {A, B, C, D, Ep, F, G};
string[] labels = {"$A$", "$B$", "$C$", "$D$", "$E$", "$F$", "$G$"};
for (int i = 0; i < points.length; ++i)
  {
    pair P = points [i];
    dot (labels [i] + (string) P, T * P, (P.y == A.x + s ? S : E));
  }
\end{verbatim}

\subsection*{Tehtävät}
\label{sec:org034a637}
\begin{enumerate}
\item Piirrä \(800\times 600\)-kokoiseen piirtoikkunaan 
a) nelikulmio, jonka vasen yläkulma on pisteessä \((160, 90)\) ja
   jonka leveys on 200 ja korkeus 300
b) kolmio, jonka kärkipisteet ovat \((110, 500)\), \((370, 420)\)
   sekä \((480, 570)\)
c) nelikulmio, jonka kärkipisteet ovat vastapäivään kiertäen
   lueteltuina \((730, 50)\), \((450, 370)\), \((770, 530)\)
   sekä \((610, 250)\).
\item Piirrä seuraavanlaiset kuvat. Piirtoikkunan koko on \(200\times 200.\)
a) 
\begin{verbatim}
size (200, 200);
triangle (0, 100, 0, 200, 200, 0);
\end{verbatim}
b) 
\begin{verbatim}
size (200, 200);
quad (0, 0, 200, 200, 200, 100, 100, 0);
triangle (0, 100, 0, 200, 100, 200);
\end{verbatim}
c) 
\begin{verbatim}
size (200, 200);
quad (0, 100, 100, 200, 200, 100, 100, 0);
rect (50, 50, 100, 100);
\end{verbatim}
\item Piirrä haluamasi kokoiseen piirtoikkunaan valitsemasi kokoinen
a) puolisuunnikas
b) tasakylkinen kolmio, joka ei ole suorakulmainen
c) suunnikas, joka ei ole suorakulmio.
\item Piirrä seuraavat kuviot. Piirtoikkunan koko on \(300\times 200.\)
a) Suorakaide on piirtoikkunan keskellä, ja suorakaiteen leveys
   ja korkeus ovat puolet piirtoikkunan vastaavista mitoista.
\begin{verbatim}
size (300, 200);
rect (75, 50, 150, 100);
\end{verbatim}
b) Ulomman suunnikkaan kaksi kärkipistettä jakavat piirtoikkunan
   pidemmät sivut suhteessa \(1:2.\) Sisemmän suunnikkaan
   kärkipisteet ovat ulomman suunnikkaan sivujen keskipisteitä.
\begin{verbatim}
size (300, 200);
quad (100, 0, 300, 100, 200, 200, 0, 100);
quad (50, 50, 200, 50, 250, 150, 100, 150);
\end{verbatim}
\item Piirrä haluamasi näköiset ja kokoiset versiot isoista kirjaimista
T, X, A, B sekä Q. Alla esimerkkinä yksi versio
A-kirjaimesta.
\begin{verbatim}
size (300, 200);
rect (75, 100, 150, 30);
quad (250, 200, 300, 200, 175, 0, 125, 0);
quad (0, 200, 50, 200, 175, 0, 125, 0);
\end{verbatim}
\item Tämän tehtävän voit tehdä vain, mikäli olet jo opiskellut
\emph{Pythagoraan lauseen}. Piirrä tasasivuinen kolmio, jonka sivun
pituus on 120. Piirrä kolmio \(300\times 300\)-kokoisen
piirtoikkunan keskelle siten, että kolmion sivuille jää yhtä
paljon tilaa ja myös ylä- ja alapuolelle jää yhtä paljon
tilaa. Voit pyöristää laskujesi tuloksina saamasi koordinaatit
kokonaisluvuiksi.
\end{enumerate}
\subsection*{Ratkaisuja}
\label{sec:org1c0fa1e}
Saman asian tekevä tietokoneohjelma voidaan kirjoittaa monella eri
tavalla. Nämä ratkaisut ovat vain esimerkkejä.
\begin{enumerate}
\item \begin{verbatim}
size (800, 600);
rect (160, 90, 200, 300);
triangle (110, 500, 370, 420, 480, 570);
quad (730, 50, 450, 370, 770, 530, 610, 250);
\end{verbatim}
\item a) 
\begin{verbatim}
size (200, 200);
triangle (0, 100, 0, 200, 200, 0);
\end{verbatim}
b) 
\begin{verbatim}
size (200, 200);
quad (0, 0, 200, 200, 200, 100, 100, 0);
triangle (0, 100, 0, 200, 100, 200);
\end{verbatim}
c) 
\begin{verbatim}
size (200, 200);
quad (0, 100, 100, 200, 200, 100, 100, 0);
rect (50, 50, 100, 100);
\end{verbatim}
\item Esimerkkejä mahdollisista ratkaisuista.
a) 
\begin{verbatim}
size (300, 200);
quad (50, 50, 130, 50, 190, 150, 10, 150);
\end{verbatim}
b) 
\begin{verbatim}
size (300, 200);
triangle (50, 150, 250, 150, 150, 100);
\end{verbatim}
c) 
\begin{verbatim}
size (300, 200);
quad (50, 150, 200, 150, 250, 50, 100, 50);
\end{verbatim}
\item a) 
\begin{verbatim}
size (300, 200);
rect (75, 50, 150, 100);
\end{verbatim}
b) 
\begin{verbatim}
size (300, 200);
quad (100, 0, 300, 100, 200, 200, 0, 100);
quad (50, 50, 200, 50, 250, 150, 100, 150);
\end{verbatim}
\item Useita eri ratkaisuja.
\item Tasasivuisen kolmion korkeudeksi saadaan Pythagoraan lauseella
noin 104.
\begin{verbatim}
size (300, 300);
triangle (90, 202, 210, 202, 150, 98);
\end{verbatim}
\end{enumerate}
\end{document}
